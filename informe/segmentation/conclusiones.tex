\section{Conclusión y trabajos futuros}

A lo largo del trabajo se observ\'o que el m\'etodo propuesto es considerablemente eficiente en lo que respecta al tiempo de c\'omputo, pero no se pudo encontrar una aplicaci\'on especifica en la que su desempe\~no como separador de objetos resulte útil. De todas formas se pudo comprobar experimentalmente que lo propuesto en \cite{Felzenszwalb2004} se cumpl\'ia y que la m\'etrica propuesta en este trabajo reflejaba suficientemente bien lo planteado en \cite{li2013benchmark}. \\
\indent Algunas de las cosas sobre las que se podr\'ia trabajar a futuro son considerar una nueva funci\'on de \textit{threshold} que proporcione mejores resultados en cuanto a la segmentaci\'on y evaluar si el tiempo de c\'omputo resultante sigue siendo bueno o por lo menos aceptable. También resta encontrar alguna aplicaci\'on en donde la segmentaci\'on sea lo suficientemente buena para resolver un problema m\'as complejo. 