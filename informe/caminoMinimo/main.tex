\documentclass[10pt,a4paper]{article}
\setlength{\parskip}{0.8em}

\usepackage[utf8]{inputenc} % para poder usar tildes en archivos UTF-8
\usepackage[spanish]{babel} % para que comandos como \today den el resultado en castellano
\usepackage[left=2cm, right=2cm, top=2cm, bottom=2cm]{geometry}
\usepackage{caratula}

\usepackage{amsmath,amsfonts,amssymb,mathtools,amsthm,epsfig,epstopdf,url,array}
\usepackage{xcolor}
\usepackage{xspace}
\usepackage{xargs}
\usepackage{ifthen}
\usepackage{amsmath}
 
\usepackage{enumerate}
\usepackage{multirow}
\usepackage{float}

\usepackage{verbatim}


\usepackage{caption}

% \usepackage[onelanguage, spanish]{algorithm2e}
  % \NoCaptionOfAlgo
% \LinesNumbered\RestyleAlgo{ruled}\IncMargin{1em}\DontPrintSemicolon\SetArgSty{}\SetCommentSty{textsf}\SetFuncSty{textsf}
% \SetKwProg{For}{para}{ hacer}{fin}
% \SetKwProg{Fn}{función}{:}{fin}

\def\code#1{\texttt{#1}}

\usepackage{mathtools} 
\usepackage{changepage}
\usepackage{pdfpages}
\usepackage{hyperref}
\usepackage[page, toc]{appendix}
\usepackage[nottoc]{tocbibind}
\usepackage{subfigure}
\usepackage{graphicx}
\usepackage{euler}
\usepackage{listings}
\usepackage{tikz}
\usetikzlibrary{graphs,graphs.standard}
\usetikzlibrary{arrows}
\usepackage{algorithmicx, algpseudocode, algorithm}
\usepackage{float}

\newcommand{\norm}[1]{\left\lVert#1\right\rVert}
\newtheorem{theorem}{Theorem}[section]
\newtheorem{corollary}{Corollary}[theorem]
\newtheorem{lemma}[theorem]{Lemma}

\DeclareMathOperator{\tc}{\textbf{tc}}

\begin{document}

\titulo{Trabajo Práctico 2: Llenalo con super}

\materia{Algoritmos y estructuras de datos III}

\integrante{Martino, Maximiliano}{123/17}{maxii.martino@gmail.com}
\integrante{Blufstein, Marcos}{300/17}{mjblufstein@dc.uba.ar}
\integrante{Peretti, Olivia}{359/17}{operetti@dc.uba.ar}
\integrante{Pironio, Nicolás}{37/17}{npironio@dc.uba.ar}
\maketitle

\clearpage
\tableofcontents

\clearpage

\section{Introducci\'on}

En el presente trabajo se aborda el problema de segmentaci\'on de im\'agenes: dada una imagen buscamos obtener una partici\'on de esta donde cada regi\'on delimitada represente un objeto identificable y distinguible. Claramente este no es un problema bien definido, ya que depende de la percepci\'on subjetiva de cada individuo sobre que es lo que se puede observar en una imagen espec\'ifica. Debido a esto es com\'un que distintos trabajos busquen dar soluciones a conjuntos de fotos sobre un área determinada, como por ejemplo im\'agenes m\'edicas para el estudio de tumores en el cerebro, aprovechando características especiales del objeto de estudio. \\

\begin{figure}[H]
	\begin{center}
		\subfigure[Imagen original]{\includegraphics[height=208pt, width=156pt]{segmentaciones/museo.jpg}}
		\subfigure[Segmentaci\'on generada ($k=12500$)]{\includegraphics[height=208pt, width=156pt]{segmentaciones/museo_seg.jpg}}
	\end{center}
	\caption{Ejemplo de una segmentaci\'on}
	\label{ejemplo}
\end{figure} 

Se puede observar en la figura \ref{ejemplo} que si bien se logran distinguir las partes centrales de la im\'agen (persona, cuadrados, cuadros, pared, piso) en la imagen original pueden haber muchos factores que influencian de manera negativa en la separaci\'on. Un claro ejemplo es la luz de tubo junto a la puerta que impacta notablemente sobre el piso, o la sombra del sujeto afectando sobre los cuadrados. Al mismo tiempo, como se explicara m\'as adelante, buscar una segmentaci\'on muy general puede perder detalles que podr\'ian ser interesantes, como las distinciones entre cuadrados dentro del cuadrado principal en el piso.\\ 
Para la resoluci\'on del problema, el trabajo se basa en el m\'etodo expuesto por Pedro F. Felzenszwalb y Daniel P. Huttenlocher \cite{Felzenszwalb2004} que, como describiremos en las secciones siguientes, proponen un algoritmo eficiente basado en grafos para conseguir una segmentaci\'on ``ni muy fina ni muy gruesa'' en su terminolog\'ia. A lo largo del trabajo se estudia este m\'etodo en profundidad, considerando las distintas implementaciones posibles, midiendo la eficiencia de estas e intentando responder a la pregunta de si el predicado propuesto por Felzenszwalb y Huttenlocher verdaderamente se refleja en los resultados obtenidos.


\clearpage

\section{Desarrollo}

A continuaci\'on presentaremos las ideas generales e intuiciones que se desarrollan en \cite{Felzenszwalb2004} y discutiremos sobre las implementaciones posibles.\\
\indent Como se mencion\'o, el m\'etodo se basa en modelar el problema con un grafo. Este se deduce de la imagen de la siguiente manera: $G=(V,\ E)$ donde 
\begin{itemize}
	\item[] $V$: píxeles de la imagen.
	\item[] $(v_i,\ v_j)\in E$: representa la disimiltud entre el pixel $v_i$ y $v_j$.
\end{itemize}
 
\indent De esta forma podemos definir que una segmentaci\'on de la imagen es tomar un \textit{subgrafo inducido} de $G$ o dicho de otra forma una segmentaci\'on $S$ es una partici\'on en componentes conexas de $V$. Esta definici\'on se justifica con la idea que los vértices de una componente deben ser similares entre sí, y no as\'i con los de una componente distinta. Esto se termina traduciendo en que los ejes de vértices en la misma componente deben tener un peso ``chico'' mientras que entre vértices de componentes distintas deben tener un peso ``grande''.\\
\indent A partir de este modelo se construye un predicado que busca responder a la pregunta de si hay evidencia de que existe un l\'imite entre dos componentes distintas. Este se basa en comparar la diferencia entre dos vértices vecinos de dos componentes distintas, teniendo en cuenta la diferencia inter-componente de cada uno para reflejar el valor local de cada componente.\\
\indent Se trabajan con dos conceptos para poder definir este predicado. Por una parte se define para una componente $C$ su \textit{internal difference} (diferencia interna) $Int(C)$ como el mayor de los pesos de las aristas presentes en el \textit{árbol generador m\'inimo} (AGM) de $C$. La intuici\'on que refleja esta medida es que la componente admite que los píxeles tengan a lo sumo una disimilaridad de $Int(C)$.\\
\indent Luego para componentes $C_1, \ C_2$ definimos la diferencia entre ellos $Diff(C_1,\ C_2)$  como la arista de menor peso entre $v_1\in C_1$ y $v_2 \in C_2$. En un principio esta última no refleja substancialmente la relaci\'on entre dos componentes, ya que tiene en cuenta una única ``conexi\'on'', pero se observa empíricamente que los resultados obtenidos son razonables y que de esta medida se desprende un algoritmo considerablemente eficiente.\\
\indent Dichas estas definiciones se explica el predicado $D$ como:
\[
	D(C_1,\ C_2)=\begin{cases}
               		True            & \text{si } Dif(C_1, \ C_2)>MInt(C_1,\ C_2)\\
               		False			& \text{Caso contrario}
           		 \end{cases}
\]
con $MInt(C_1, \ C_2)=Min(Int(C_1) + \tau(C_1), Int(C_2) + \tau(C_2))$ con $\tau$ una funci\'on para regular el grado en el cual la diferencia entre componentes tiene que ser mayor que la diferencia interna para afirmar que existe un l\'imite entre componentes. Intuitivamente si solo se tomase el m\'inimo entre diferencias internas no reflejar\'ia la mayor parte de la informaci\'on interna de la componente. Con esto en mente, podemos tomar $\tau(C)=\frac{k}{\#C}$ con $k$ un hiperparámetro. De esta forma la elecci\'on de $k$ influye sobre cuándo se decide que hay evidencia para definir que hay un l\'imite entre componentes, siendo que un $k$ mayor es propenso a resultar en delimitar menos componentes y un $k$ más chico implicar\'ia que para afirmar que hay un l\'imite para una componente peque\~na se requiere m\'as evidencia. \\
\indent Considerando estas nociones se desprende el siguiente algoritmo para computar la segmentaci\'on:
\begin{enumerate}
	\item Ordenar no decrecientemente las aristas: $E=(e_1 \dots e_m)$.
	\item Empezar con la partici\'on $S_0$ con cada vértice en una componente separada.
	\item Para $i=1\dots m$ hacer 4.
	\item Sea $C_1$ la componente de la cola de $e_i$ y $C_2$ la componente de la cabeza de $e_i$. Si $C_1\neq C_2$ y $peso(e_i) < MInt(C_1,\ C_2)$ entonces $S_i$ se construye uniendo las componentes $C_1$ y $C_2$ de $S_{i-1}$.
	\item Devolver $S=S_m$.
\end{enumerate}

Notar que el algoritmo que se obtiene de las definiciones es en esencia el algoritmo de Kruskal para obtener un AGM. Recordemos que el invariante de Kruskal es mantener un bosque generador m\'inimo y en cada iteraci\'on agregar una arista segura y candidata. La diferencia que podemos encontrar en el algoritmo de segmentaci\'on propuesto es lo que definimos como arista candidata: en nuestro caso una arista es candidata cuando es la m\'inima que une dos componentes distintos y además su peso no supera $MInt$. Esto claramente no tiene porqué resultar en un AGM pero sí en un bosque generador m\'inimo donde sus componentes s\'i son AGM, dándole una nueva interpretaci\'on a lo previamente desarrollado. \\
\indent Si bien no es el enfoque de este trabajo demostrar las propiedades que posee el algoritmo si vamos a mencionarlas:\\
Se define que una segmentaci\'on es \textit{muy fina} si existen regiones $C_1$, $C_2$ para las cuales no existe evidencia de que haya un l\'imite entre ellas seg\'un el predicado $D$. \\
\indent Se dice que $T$ es un \textit{refinamiento propio} de una segmentaci\'on $S$ si $\forall C'\in T, \ \exists C\in S \text{ tal que } C'\subsetneq C$ \\
\indent Una segmentaci\'on $S$ es \textit{muy gruesa} si existe un refinamiento propio de $S$ que no es muy fino.\\
\indent Con estas nociones establecidas se prueba que el algoritmo propuesto provee una segmentaci\'on que no es ni muy fina ni muy gruesa.
\clearpage

\section{Implementaci\'on}
En esta sección explicaremos las estructuras elegidas para representar los datos.

\subsection{Arco}
Para representar los ejes de un grafo creamos la struct arco, con los siguientes atributos:
\begin{itemize}
\item peso
\item cola
\item cabeza
\end{itemize}
Si tenemos una arista m que va de un nodo $a$, a un nodo $b$ con un valor $k$. Luego podemos representar m de la siguiente forma:
\begin{itemize}
\item peso = k
\item cola = a
\item cabeza = b
\end{itemize}

\subsection{Grafo}
Para representar grafos elegimos una representación basada en lista de adyacencias. Para esto, creamos la estructura grafoAd con los siguientes atributos:
\begin{itemize}
\item listaAd: es un vector de listas. Cada posición $i$ del vector guarda una lista con los arcos salientes del nodo $i$. Es decir, guarda los vecinos a los que se puede acceder desde $i$, con el peso de la arista que lo permite\footnote{Además por ser eje guarda la cola de la arista, que en este caso será $i$ para todos los ejes de la lista.}.
\item cantNodos: es un entero que guarda la cantidad de nodos del grafo.
\item cantAristas: es un entero que guarda la cantidad total de aristas del grafo.
\end{itemize}

\indent Y los siguientes métodos:
\begin{itemize}
\item grafoAd: es el constructor de la clase. Inicializa cantNodos y cantAristas en cero, y la lista de adyacencias como un vector vacío.
\item agregarNodo: agrega una nueva posición al vector, con una lista vacía.
\item agregarEje: agrega una arista al grafo, con el peso y los nodos especificados en los parámetros.
\end{itemize}

\newpage

\subsection{Lectura de la entrada}

A partir de la entrada del programa, generamos el grafo modificado. Para lograrlo, generamos un grafo con n'=61*n nodos, donde el nodo i representa al nodo i/61 del grafo original, en un estado con i mod 61 Litros de nafta. Por ejemplo, si teníamos un grafo con 100 nodos, en el grafo modificado el nodo 176 representa a la ciudad 2 con 54 litros en el tanque. \\
\indent Sabiendo esto, para cada costo $C_{i}$ de combustible leido generamos 61 vertices, donde cada uno está unido al siguiente con una arista de costo $C_{i}$. De esta manera, logramos generar todas las aristas de tipo A. \\
\indent Por último, para cada ruta leida, generamos las aristas de tipo B. Si una ruta es representada con los enteros $a_{i}$, $b_{i}$, $l_{i}$ (ambas ciudades seguidas del costo de la ruta), vamos a decir que una arista representa un movimiento valido si $a_{i}$ representa un estado con $K$ litros en el tanque, y $b_{i}$ representa un estado con $k$-$l_{i}$ litros. Luego, simplemente debemos conectar un nodo de la ciudad $a_{i}$ a otro de $b_{i}$ (y viceversa) si representan un movimiento valido. \\
\indent Presentamos el pseudocodigo a continuación:

\begin{algorithm}[H]
\caption{crear grafo modificado}
\label{img2sorted}
\begin{algorithmic}[1]
\Procedure{crearGrafoModificado}{}
\State $i \gets 0$
\For{$i < cantidadDeCiudades$}
	\State G.agregarNodo()
	\State $j \gets 1$
	\State precioNafta $\gets$ precioNaftaEn(i)
	\For{$j < 61$} \Comment{Generamos aristas de tipo A}
		\State G.agregarNodo()			\Comment{El nodo de estar en la ciudad i con j litros}
		\State a $\gets 61*i+j-1$			\Comment{Tener j-1 litros en la ciudad i}
		\State b $\gets$ siguienteDe(a)		\Comment{Tener j litros en la ciudad i}
		\State G.agregarEje(precioNafta, a, b) \Comment{Costo de pasar de tener j-1 litros, a tener j en la ciudad i}
	\EndFor
\EndFor

\State $i \gets 0$
\For{$i < cantidadDeRutas$}
	\State a $\gets$ extremoADeArista(i)
	\State b $\gets$ extremoBDeArista(i)
	\State l $\gets$ cantidadDeLitrosDeAaB(i)
	\State j $\gets$ 0
	\For{j $<$ 61} 					\Comment{Generamos aristas de tipo B}
		\If{j-l $\geq$ 0} 				\Comment{Si puedo ir de a a b con la nafta que tengo}
			\State a' $\gets$ a*61+j
			\State b' $\gets$ b*61+j-l
			\State G.agregarEje(0, a', b') 	\Comment{La arista de viajar de la ciudad a a b con l litros de nafta}
			\State a' $\gets$ a*61+j-l
			\State b' $\gets$ b*61
			\State G.agregarEje(0, b', a')	 \Comment{La arista de viajar de la ciudad b a a con l litros de nafta}
		\EndIf
	\EndFor
\EndFor
\State return G
\EndProcedure
\end{algorithmic}
\end{algorithm}

\indent Complejidad: $\mathcal{O}(n+m)$

\subsection{Impresión por salida estándar}
Para imprimir el resultado, simplemente hay que ser cuidadosos y recordar que nuestro nuevo grafo contiene nodos que no nos interesan (Solo queremos empezar y terminar en estados en los que tenemos 0 Litros en el tanque). De esta forma, si terminamos con una matriz de distancias, solo debemos imprimir aquellas posiciones cuyas columnas o filas sean iguales a 0 modulo 61.

%\begin{algorithm}[H]
%\caption{imprimir resultados del programa}
%\label{}
%\begin{algorithmic}[1]
%\Procedure{imprimirResultado}{$matriz$ distancias}
%\State i $\gets$ 0
%\For{i$<$distancias.size()}
%	\State j $\gets$ 0
%	\For{j$<$distancias.size()}
%		\If{i $\neq$ j}
%			\State imprimir(i, j, distancias$[i][j]$) \Comment{Imprimir la mínima distancia de ir de i a j}
%		\EndIf
%	\EndFor
%\EndFor
%\EndProcedure
%\end{algorithmic}
%\end{algorithm}
%
%\indent Complejidad: $\mathcal{O}()$

\clearpage

\section{Experimentación}

\subsection{Variación en la cantidad de nodos}
En esta experimentación, generamos grafos de la siguiente manera: \\
\begin{itemize}
\item n fue variando entre 10, 100, 200, 300, 400 y 500.
\item m siempre mantuvo la proporción m = $\frac{n*(n-1)}{4}$
\item los costos de nafta fueron elegidos de manera aleatoria en el rango [1,100]
\item los largos de las rutas en litros fueron tomados de manera aleatoria en el rango [1,60]
\end{itemize}

La idea es ver como se comportan los distintos algoritmos, cuando vamos teniendo un grafo denso cada vez más grande. Es importante tener en cuenta que ambas versiones de Dijkstra y Bellman-Ford fueron corridos n veces (ya que el output del programa tiene todas las distancias posibles) mientras que Floyd se corre una única vez. \\
Los resultados obtenidos fueron los siguientes:
\begin{figure}[H]
   \begin{minipage}{0.5\textwidth}
     \centering
     \includegraphics[width=1\linewidth]{img/exp1_2.png}
     \caption{Comparación del problema con los 4 algoritmos}
   \end{minipage}\hfill
   \begin{minipage}{0.5\textwidth}
     \centering
     \includegraphics[width=1\linewidth]{img/exp1_1.png}
     \caption{Comparación del problema quitando el algoritmo de Floyd}
   \end{minipage}
\end{figure}

Como podemos observar, el algoritmo de Floyd crece de manera mucho más acelerada que los otros algoritmos, a tal punto que no pudo ser corrido para los casos más grandes. Esto puede deberse a que el grafo modificado que usamos para resolver el problema tiene 61*n nodos, por lo que la matríz que debe resolver Floyd resulta muy grande. De esta manera, está calculando las distancias mínimas tomando como origen a cualquier estado (cantidad de litros iniciales) de una ciudad, cuando sólamente nos importa tomar el estado del tanque vacío como origen. Esto no afecta tan marcadamente a los otros algoritmos, ya que únicamente los corremos con dicho estado como origen. \\
\indent En este caso, sabemos que m=$\mathcal{O}(n^{2})$, por lo que las complejidades de cada algoritmo corrido n veces son:
\begin{itemize}
\item Dijkstra: $\mathcal{O}(n^{3})$
\item Dijkstra con cola de prioridad: $\mathcal{O}(n^{3}*lgn)$
\item Bellman-Ford: $\mathcal{O}(n^{4})$
\item Floyd: $\mathcal{O}(n^{3})$
\end{itemize}

Como se ve en el gráfico, todas las curvas tienen un crecimiento polinomial, como se esperaba. Lo que quizas es mas inusual, es el comportamiento del algoritmo de dijkstra sin cola de prioridad. En estos casos, se esperaría que tenga una mejor performance que los demás, pero vemos que tiene un crecimiento considerablemente más acelerado. Esto probablemente se deba a que, aunque m=$\mathcal{O}(n^{2})$, en el grafo modificado la constante oculta sea lo suficientemente pequeña como para que $n^{2}$ sea considerablemente peor a m en la práctica.

\subsection{Variación en la proporción de aristas}
En esta experimentación, generamos grafos de la siguiente manera: \\
\begin{itemize}
\item n fue fijado en 100.
\item m fue variando entre 10, 100, 500, 1000, 1500, 2000, 3000, 4000
\item los costos de nafta fueron elegidos de manera aleatoria en el rango [1,100]
\item los largos de las rutas en litros fueron tomados de manera aleatoria en el rango [1,60]
\end{itemize}

En este experimento analizamos que sucede si, dado un conjunto fijo de ciudades, vamos construyendo nuevas rutas. Al igual que en el experimento anterior, todos los algoritmos salvo el de Floyd fueron corridos n=100 veces. \\
Veamos los resultados:
\begin{figure}[H]
   \begin{minipage}{0.5\textwidth}
     \centering
     \includegraphics[width=1\linewidth]{img/exp2_2.png}
     \caption{Comparación del problema con los 4 algoritmos}
   \end{minipage}\hfill
   \begin{minipage}{0.5\textwidth}
     \centering
     \includegraphics[width=1\linewidth]{img/exp2_1.png}
     \caption{Comparación del problema quitando el algoritmo de Floyd}
   \end{minipage}
\end{figure}

Nuevamente, el algoritmo de Floyd fue mucho más lento que los demás. Sin embargo, podemos ver que en este caso el tiempo de computo de Floyd se mantiene constante. Esto era de esperarse, pues la complejidad del mismo depende únicamente de n, que fue fijado en 100.
En cuanto a los otros algoritmos, asumiendo que en este caso n es una constante, las complejidades teóricas en cada caso son:
\begin{itemize}
\item Dijkstra: $\mathcal{O}(1)$
\item Dijkstra con cola de prioridad: $\mathcal{O}(m)$
\item Bellman-Ford: $\mathcal{O}(m)$
\end{itemize}
Sin embargo, es importante notar que existe un $\mathcal{O}(m)$ en la complejidad del algoritmo de Dijkstra sin cola de prioridad, ya que se recorren todas las aristas. Esto explica el crecimiento de su curva en el gráfico. En cuanto a Bellman-Ford y Dijkstra con cola de prioridad, si recordamos que el m va acompañado de un n y de un log(n) respectivamente, podemos explicar por qué difieren las pendientes de sus rectas.

\clearpage

\section{Conclusión y trabajos futuros}

En los resultados de los experimentos, notamos que en la mayoría de los casos obtenemos mejores tiempos utilizando el algoritmo de Dijkstra con cola de prioridad. Hubo casos en los que tanto Bellman-Ford como la otra versión de Dijkstra obtuvieron mejores resultados. Sin embargo, en ambos experimentos, su crecimiento asintótico parece ser mucho menor.
En cuanto a posibles mejoras para solucionar el problema, en un comienzo consideramos una estrategia greedy para solucionarlo. El mismo consistía en completar el grafo de entrada (si dos ciudades no estan directamente conectadas, agregamos una arista entre ellas con peso igual al camino mínimo entre ambas, medido en litros. Si dicho costo era mayor a la capacidad del tanque, el peso sería infinito), y en el hecho de que, dado un camino fijo de una ciudad A a otra ciudad B, hay una estrategia óptima para cargar combustible: \\
\indent Si $s = u_{1}, u_{2}, ... , u_{l}$ representa las paradas de recarga de un posible camino, la siguiente es una estrategia óptima para decidir la cantidad de gas que se va a llenar en cada parada:
en la parada $u_{l}$ cargamos nafta suficiente para alcanzar al nodo destino con el tanque vacío; para $j <l$:
\begin{enumerate}
 	\item Si $c(u_{j}) < c(u_{j+1})$, entonces en $u_{j}$ llenamos el tanque.
 	\item Si $c(u_{j}) \geq c(u_{j+1})$, entonces en $u_{j}$ llenamos lo suficiente para llegar a $u_{j+1}$
\end{enumerate}

Sin embargo, hubo casos que nos trajeron problemas al utilizar este método, como por ejemplo el grafo mostrado en la introducción, ya que terminabamos con caminos no simples. Pensamos en posibles soluciones a éste problema, pero complicaban considerablemente la implementación, por lo que decidimos abandonar la estrategia por la presentada en el trabajo.

\clearpage


\end{document}
